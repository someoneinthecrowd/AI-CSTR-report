\documentclass[fleqn,10pt]{wlscirep}
\usepackage[utf8]{inputenc}
\usepackage[T1]{fontenc}
\usepackage{bm}
\title{CTSR Autonomous Fault detection with Neural Network}

\author[1,*]{Yong Dou(yd2373)}

\affil[1]{Columbia University, Department of Chemical Engineering ,New York, NY}

\affil[*]{e-mail:yd2373@columbia.edu}


\begin{abstract}

\end{abstract}
\begin{document}

\flushbottom
\maketitle

\thispagestyle{empty}

\noindent \textbf{Key points:} Neural Network, Knowledge based system, CSTR, Fault detection}


\section*{Article types in Nature Reviews Physics}
\subsection*{Introduction}
nonisothermal \cite{bruns1977nonlinear}




\subsection*{Perspective}
There are some hidden paramereeris that are still very important 

\subsection*{CSTR system}
we assume this is a first order exothermic reaction
\begin{equation}
    q(C_{inlet}-C_{reaction})=C_A Vk(T)
\end{equation}
\begin{equation}
    q \rho C_p(T_{inlet}-T_{reactor})
\end{equation}
q is the hidden parameter
A Technical Review\cite{TR} surveys the current of state-of-the-art capabilities in a certain area. Technical Reviews provide accessible summaries of the state-of-the-art figures of merit for techniques, devices and/or materials; comparisons of different methods with an overview of their applicability; comparisons of scientific software codes for specific applications; guidelines for data analysis of specific datasets and in particular large data releases from big instruments.

Technical Reviews are approximately 5,000–-6,000 words long and typically include 8 display items and 150 references. 
\subsection*{Expert Recommendation}

Expert Recommendations are collective opinion pieces, authored by panels of specialists, that present the outcome of an analysis or discussion, and suggest a course of action, best scientific practices or methodological guidelines.

Expert Recommendations are typically 1,500–-2,000 words long and may include up to 4 display items and 50 references.

\bibliography{sample}

\noindent \textbf{For Reviews only, highlighted references (optional)} Please select 5–-10 key references and provide a single sentence for each, highlighting the significance of the work.

\section*{Acknowledgements (optional)}
The authors thank Erwin the Cat for useful discussions. Please edit as necessary.

\section*{Author contributions}
The authors contributed equally to all aspects of the article. Please edit as necessary. Note that the information must be the same as in our manuscript tracking system.

\section*{Competing interests}
The authors declare no competing interests. Please edit as necessary. Note that the information must be the same as in our manuscript tracking system.

\section*{Publisher’s note}
Springer Nature remains neutral with regard to jurisdictional claims in published maps and institutional affiliations.

\section*{Supplementary information (optional)}
If your article requires supplementary information, please include these files for peer-review. Please note that supplementary information will not be edited.

\newpage
\section*{Box 1 (Optional)}
This is a Box, which can contain a figure, and which should have no more than 300 words of text.

\begin{figure}[ht]
\centering
\includegraphics[width=\linewidth]{fig}
\caption{The figure caption should start with a title explaining the figure. Figures should be self-consistent so please redefine all acronyms and define all symbols. Example: GHZ, Greenberger–Horne–Zeilinger, OAM, orbital angular momentum. Please provide credit lines for panels reproduced from the literature. Example: Panels a and b are reproduced from Ref. \cite{TR}.}
\label{fig}
\end{figure}

\begin{table}[ht]
\centering
\begin{tabular}{|l|l|l|}
\hline
Particle & Mass & Charge \\
\hline
Electron & $9.10938356(11)\times10^{-31}$ kg & $-1e$ \\
\hline
Proton & $1.672621898(21)\times10^{-27}$ kg & $+1e$ \\
\hline
Neutron & $1.674927471(21)\times10^{-27}$ kg & $0$ \\
\hline
\end{tabular}
\caption{\label{tab}Tables have titles but no captions are allowed. However, all symbols and acronyms used in a table should be defined in a footnote. Example: Here $e$ is the elementary charge.}
\end{table}

\section*{Glossary terms (optional)}
It is possible  to include glossary terms to define some technical terms. For each glossary term, please provide a short (maximum 30 word) definition. Please list these in order of first appearance in the text. In the published version, they will appear in the margins of the document. 

Example: \\
\textbf{Lam\'e parameters:} A possible pair of parameters that characterize the Cauchy elasticity tensor in an isotropic homogeneous medium. The second Lam\'e parameter is identical to the shear modulus.



\end{document}